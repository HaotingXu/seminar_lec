\documentclass[CJK]{beamer}
\input{macros.tex}
\usepackage{amssymb}
\newcommand{\field}{\mathscr{F}}

\newcommand{\reals}{\mathbb{R}}
\newcommand{\complexs}{\mathbb{C}}
\newcommand{\ints}{\mathbb{Z}}
%\newcommand{\dim}{\mathrm{dim\ }}
\newcommand{\diag}{\mathrm{diag \ }}
\newcommand{\up}{\uparrow}
\newcommand{\down}{\downarrow}
\newcommand{\su}{\mathfrak{su}}
\newcommand{\so}{\mathfrak{so}}
\newcommand{\tr}{\mathrm{tr\ }}
\newcommand{\card}{\mathrm{card \ }}

\newtheorem{thm}{定理}
\newtheorem{axm}{公理}
\newtheorem{dfn}{定义}

%\cpic{<尺寸>}{<文件名>}}用于生成居中的图片。
\newcommand{\cpic}[2]{
\begin{center}
\includegraphics[scale=#1]{#2}
\end{center}
}

%\cpicn{<尺寸>}{<文件名>}{<注释>}用于生成居中且带有注释的图片,其label为图片名。
\newcommand{\cpicn}[3]
{
\begin{figure}[h!]
\cpic{#1}{#2}
\caption{#3\label{#2}}
\end{figure}
}

\title{DG - Riemann Curvature Tensor}
  \author{}
  \date{}


\begin{document}

\begin{frame}
 
\begin{center}
\begin{Large}
  \bch
  \begin{center}
\includegraphics[width = 1.2in]{cover}
\end{center}

{\bf D}iffrentiable {\bf G}eometry

{\vskip 0.1in}

Riemann Curvature Tensor

\ech
\end{Large}
\end{center}


\vskip 0.1in
\begin{center}
Haoting Xu
\vskip 0.1in
xuht9@mail2.sysu.edu.cn
\vskip 0.1in
{\tiny \url{https://github.com/HaotingXu/seminar_lec/}}\\
\end{center}


\end{frame}
\section{Abstract Index Notion}
\begin{frame}\frametitle{\bch抽象指标记号\ech}
  \bch
  回顾(k,l)型张量的定义。现在将张量记为$T^{ab}_{\spa\spa c}$,这种记号只表示张量类型,并不代表张量的分量。因此当这种记号出现时,浮现在脑子里的只能是映射。
  \ech
\end{frame}
\begin{frame}\frametitle{\bch 抽象指标记号\ech}
  \bch
  用希腊符号表示张量的分量,例如$(2,1)$型张量的分量记为$T^{\mu \nu}_{\spa\spa\sigma}$,这又被叫做具体指标,因为张量的分量真的就是一堆数字,所以和以前线性代数里边的指标一样,满足爱因斯坦求和规则。现在考虑将张量用张量的基底展开,有
  \be
  T^{ab}_{\spa\spa c} = T^{\mu\nu}_{\spa\spa\sigma} (\partial_\mu)^a (\partial_\nu)^b (dx^\sigma)_c
  \ee
  考虑张量的缩并,例如$T$的第二个上指标和第一个下指标缩并,就方便的记为$T^{ab}_{\spa \spa b }$,因此我们说,对于抽象记号,重复指标代表缩并,而并不代表“求和”。特别地,考虑一个一般的张量(以$(2,1)$型张量为例)和一个矢量缩并,并利用上面的展开公式和缩并的定义
  \be
  T^{ab}_{\spa\spa  c}v^c = T^{\mu\nu}_{\spa\spa\sigma} (\partial_\mu)^a (\partial_\nu)^b (dx^\sigma)_c(\partial_\delta) \otimes v^\rho (\partial_\rho)^c(dx^\delta)  =T (\cdot,\cdot,v) 
  \ee
  因此,{\color{blue} 将一个矢量和张量缩并,就相当于将矢量传进去}。请读者证明,对于余切矢量也是这样。
  \ech
\end{frame}
\begin{frame}\frametitle{\bch抽象指标记号\ech}
  \bch
  回顾张量分量的定义,自然有
  \be
  {\color{blue} T^{\mu\nu}_{\spa \spa\sigma} = T(dx^\mu,dx^\nu; \partial_{\sigma}) = T^{ab}_{\spa\spa c} (dx^\mu)_a (dx^\nu)_b (\partial_\sigma)^c } 
  \ee
  \ech
\end{frame}
\begin{frame}\frametitle{\bch 思考题 \ech}
  \bch
  \begin{itemize}
  \item 使用分量和基底的形式表示度规张量$g_{ab}$
  \item 使用张量的抽象指标表示度规张量的分量$g_{\mu\nu}$
  \end{itemize}
  \ech
\end{frame}
\section{Derivitative Operator}
\begin{frame}\frametitle{\bch无挠导数算符\ech}
  \bch
  \begin{dfn}[无挠导数算符]\label{operator}
    $\mathcal{F}(k,l)$代表流形$\mathcal{M}$上全体$C^{\infty}(k,l)$型张量场的集合。定义导数算符为$\nabla:\mathcal{F}_{\mathcal{M}}(k,l)\rightarrow \mathcal{F}_{\mathcal{M}} (k,l+1)$,如果
    \begin{enumerate}[(a)] 
    \item 对张量线性
    \item Leibnitz律
    \item 与缩并可以交换
    \item $v(f) = v^a\nabla_a f = v(\nabla f) = v(df)$
      \item 无挠$\nabla_a\nabla_b f = \nabla_b \nabla_a f$
    \end{enumerate}
    \end{dfn}
  \ech
\end{frame}
\begin{frame}\frametitle{\bch 无挠导数算符的性质\ech}
  \bch
  条件(c)表明
  \be
  \nabla_a(v^b\omega_b) = v^b\nabla_a\omega_b + \omega_b\nabla_av^b
  \ee
  从上面条件(d)自然地可以得出导数算符对于标量场的作用
  \be
  \nabla_a f = (df)_a
  \ee
  可见,导数算符对于任何标量场的作用都一样。
  \ech
\end{frame}
\begin{frame}\frametitle{\bch 导数算符有多少?\ech}
  \bch
  可以证明,任意流形必定存在满足定义~\ref{operator}的导数算符。
  
  上面我们看到,导数算符作用在标量场上没有任何区别,但是我们预测,导数算符作用在更多指标的张量上一定是有区别的。那么两个导数的差别到底会是怎么样呢?我们有下面的定理
  \begin{thm}
    如果有两个$(0,1)$型张量场,它们在$p$点相等,即$\omega_b = \omega^{\prime}_b = \mu_b$,那么对$\mathcal{M}$上任意两个导数算符$\nabla_a,\tilde{\nabla}_a$,有
    \be
    \left[ (\tilde{\nabla}_a - \nabla_a)\omega_b^{\prime}\right]_p = \left[(\tilde{\nabla}_a  - \nabla_a)\omega_b\right]_p
    \ee
  \end{thm}
  神奇吧\includegraphics[scale=0.3]{emoji} 
  \ech
\end{frame}
\begin{frame}\frametitle{\bch定理的证明\ech}
  \bch
  定理的证明十分简单,构造$\Omega_b = \omega^{\prime}_b -\omega_b$,则在$p$点有$\Omega |_p = 0$,更严谨的说,$\Omega_b$的坐标分量为0,于是有
  \be
  \nabla_a(\omega^{\prime}_b - \omega_b)|_p = \nabla_A\left[\Omega_\mu (dx^\mu)_b\right]_p = \left[(dx^\mu)_b\nabla_a\Omega_\mu\right]_p
  \ee
  同理,有
  \be
  \tilde{\nabla}(\omega_b^\prime - \omega_b)|_p =  \left[(dx^\mu)_b\tilde{\nabla}_a\Omega_\mu\right]_p
  \ee
  而$\Omega_\mu$相当于流形上的一个张量场,于是上面的两条式子相等,定理得证。
  \ech
\end{frame}
\begin{frame}\frametitle{\bch 导数算符的差异\ech}
  \bch
  于是我们得到重要结论,{\color{blue} 导数算符作用在$(0,1)$型张量之后的差值只与$\omega_b$在$p$点的值有关},于是可以写成一个普遍的映射形式
  \be
  \nabla_a\omega_b = \tilde{\nabla_a}\omega_b - C^{c}_{\spa\spa ab} \omega_c \forall \omega_b \in \mathcal{F}(0,1)
  \ee
  利用$\nabla_a$的无挠性,将导致
  \begin{thm}
    \be
    C^c_{\spa ab} = C^c_{\spa ba}
    \ee
  \end{thm}
  \ech
\end{frame}
\begin{frame}\frametitle{\bch 下标对称的证明\ech}
  \bch
  利用无挠性,有
  \be
  \nabla_a\nabla_b f = \tilde{\nabla}_a\tilde{\nabla_b}f - C^c_{\spa ab}\nabla_cf
  \ee
  交换指标并将两式相减,有
  \be
  C^c_{\spa ab} \nabla_c f = C^c_{\spa ba} \nabla_c f
  \ee
  将上式写成分量形式即得证。
  \ech
\end{frame}
\begin{frame}\frametitle{\bch 导数算符对$(1,0)$型张量场的作用\ech}
  \bch
  \begin{thm}
    \be
    \nabla_a v^b = \tilde{\nabla}_av^b + C^{b}_{\spa ac}v^c
    \ee
  \end{thm}
  考虑$\nabla_a(\omega_bv^b)$,并利用导数算符对于余切矢量的作用,即可得证。类似地,可证明
  \be
  \nabla_a T^b_{\spa c} = \tilde{\nabla_a}T^b_{\spa c} + C^b_{\spa ad}T^{d}_{\spa c} - C^d_{\spa ac}T^b_{\spa d}
  \ee
  读者可以试着搞一个一般张量的作用公式,并试着证明它。
  \ech
\end{frame}
\begin{frame}\frametitle{\bch小结 \ech}
  \bch
  因此说,给定了一个导数算符$\nabla_a$,就可以通过$C^c_{ab}$去构建$\tilde{\nabla}$,构建的这些导数算符就是流形上全部的导数算符了。因此,我们把给定导数算符$\nabla_a$的流形记为$(M,\nabla_a)$
  \ech
  \end{frame}
\begin{frame}\frametitle{\bch 普通导数算符\ech}
  \bch
  现在考虑流形上的坐标系$\{ x^\mu\}$,我们定义普通导数算符,它定义为:张量场的普通导数的坐标分量等于该张量场的坐标分量对坐标的偏导数,即
  \be
  \partial_a T^{b}_{\spa c} = (dx^\mu)_a(\partial_\nu)^b(dx^\sigma)_c \partial_\mu T^{\nu}_{\spa \sigma}
  \ee
  我们把$\partial_a$叫做普通导数算符。
  \ech
\end{frame}
\begin{frame}\frametitle{\bch普通导数算符\ech}
  \bch
  由普通导数的定义可以得到
  \begin{itemize}
    \item
  \be
  \partial_a (\partial_\nu)^b = 0,\,\, \partial_a(dx^\mu)_b=0
  \ee
\item 更强的无挠性
  \be
  \partial_a\partial_b T^{\cdots \cdots}_{\cdots \cdots} =\partial_b\partial_a T^{\cdots \cdots}_{\cdots \cdots}
  \ee
  \end{itemize}
  由定义可以知道,$\partial_a$是依赖于坐标系的导数算符,我们把不依赖于坐标系的导数算符称作协变导数算符。
  \ech
\end{frame}
\begin{frame}\frametitle{\bch Christoffel 符号\ech}
  \bch
  \begin{dfn}[克氏符]
    如果$\partial_a$是$(M,\nabla_a)$的导数算符,体现协变导数算符和普通导数算符的张量场$C^c_{\spa ab}$称为Christoffel symbol。
  \end{dfn}
  
  \ech
\end{frame}

\begin{frame}\frametitle{\bch克氏符是张量吗\ech}
  \bch
  通过前面的讨论,我们清楚地知道克氏符是个张量。但是,很多物理学家说克氏符不是张量,因为它在坐标变换下不满足张量变换律,这究竟是怎么一回事呢?

  克氏符是满足普通导数和协变导数差异的算符,即在一个坐标系下的普通导数和协变导数差异的算符$\Gamma^a_{\spa bc}$。变换到另一个坐标系,就变成了另一个坐标系导数算符与协变导数差异的算符$\bar{\Gamma}^a_{\spa bc}$。也就是说,克氏符在坐标变换下已经不是原来那个张量了。那么两个不同的张量怎么可能满足坐标变换呢,但是如果在另外一个坐标系还看原来反映原来协变导数与原坐标系差异的克氏符,是满足坐标变换的。因此,我们经常说,克氏符是依赖于坐标系的张量。
  \ech
\end{frame}
\begin{frame}\frametitle{\bch 小练习\ech}
  \bch
  可以引入记号,使得$\partial_av^b = v^\nu_{,\mu}(dx^\mu)_a(\partial_\nu)^b,$ $\nabla_a v^b = v^{\nu}_{;\mu} (dx^\mu)_a (\partial_\nu)^b$,可以用克氏符的定义证明
  \begin{itemize}
  \item $v^\nu_{;\mu} = v^\nu_{,\mu} + \Gamma^\nu_{\spa \mu \sigma}v$
  \item $\omega_{\nu ;\mu} = \omega_{\nu ,\mu} -\Gamma^{\sigma}_{\spa \mu\nu} \omega_\sigma$
  \end{itemize}
  \ech
\end{frame}

\begin{frame}\frametitle{\bch一些定理\ech}
  \bch
  \begin{thm}
    定义1的条件(c)等价于$\nabla_a\delta^b_{\spa c}=0$
  \end{thm}
  \begin{thm}
    \be
       [u,v]^a = u^b\nabla_b v^a - v^b\nabla_b u^a,\, [u,v](f)=[u,v]^a\nabla_a f 
       \ee
       如果取普通导数算符,则分量可表示为
       \be
          [u,v]^\mu = u^\nu\partial_\nu v^\mu - v^\nu\partial_\nu u^\mu
          \ee
  \end{thm}
  
  \ech
\end{frame}

  

\end{document}
