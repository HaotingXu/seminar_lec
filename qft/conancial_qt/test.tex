\documentclass[11pt]{beamer}
\input{macros.tex}
\graphicspath{{figures/}}


\newcommand{\field}{\mathscr{F}}

\newcommand{\reals}{\mathbb{R}}
\newcommand{\complexs}{\mathbb{C}}
\newcommand{\ints}{\mathbb{Z}}
%\newcommand{\dim}{\mathrm{dim\ }}
\newcommand{\diag}{\mathrm{diag \ }}
\newcommand{\up}{\uparrow}
\newcommand{\down}{\downarrow}
\newcommand{\su}{\mathfrak{su}}
\newcommand{\so}{\mathfrak{so}}
\newcommand{\tr}{\mathrm{tr\ }}
\newcommand{\card}{\mathrm{card \ }}
\newcommand{\lag}{\mathcal{L}}
\newcommand{\hamiltonian}{\mathcal{H}}
\newcommand{\op}{\mathcal{O}}
\newtheorem{thm}{定理}[section]
\newtheorem{axm}{公理}[section]
\newtheorem{dfn}{定义}[section]
\newtheorem{experience}{经验}[section]


%\cpic{<尺寸>}{<文件名>}}用于生成居中的图片。
\newcommand{\cpic}[2]{
\begin{center}
\includegraphics[scale=#1]{#2}
\end{center}
}

%\cpicn{<尺寸>}{<文件名>}{<注释>}用于生成居中且带有注释的图片,其label为图片名。
\newcommand{\cpicn}[3]
{
\begin{figure}[h!]
\cpic{#1}{#2}
\caption{#3\label{#2}}
\end{figure}
}

\title{Quantum Field Theory\\ 正则量子化}
  \author{Haoting Xu}
  \date{\today}


\begin{document}

\begin{frame}\frametitle{相对论归一化}
我们还可以得到$2E_\vec{p} \delta^{3} (\vec{p}-\vec{q})$是洛伦兹不变的。我们还可以定义相对论归一化的态$|p\rangle = \sqrt{2E_{\vec{p}}}|\vec{p}\rangle$,他们满足
\be
\langle p|q\rangle = 2E_\vec{p} \delta^{(3)}(\vec{p}-\vec{q})
\ee
\end{frame}
\begin{frame}\frametitle{产生湮灭算符}
可以将场用产生湮灭算符展开
\be
\hat{\phi}(\vec{x}) = \int \frac{d^3p}{(2\pi)^{\frac32}}\frac{1}{\sqrt{2E_\vec{p}}}(\hat{a}_\vec{p}e^{i\vec{p}\cdot\vec{x}}+\hat{a}^\dagger_\vec{p}e^{-i\vec{p}\cdot\vec{x}})
\ee
如果我们考虑海森堡绘景
\be
\hat{\phi}(x) = \hat{U}^\dagger \hat{\phi}(\vec{x})\hat{U}=e^{i\hat{H}t} \hat{\phi}(\vec{x})e^{-i\hat{H}t}
\ee
先计算$\hat{U}^\dagger a_\vec{p} \hat{U}$
\end{frame}
\begin{frame}
利用对易关系$[H,\hat{a}_\vec{p}]=-E_\vec{p}a_\vec{p}$
和公式
\be
e^\hat{A}\hat{B}e^{-\hat{A}} = \hat{B}+ \left[\hat{A},\hat{B}\right]+ \frac{1}{2!}\left[ A,\left[\hat{A},\hat{B}\right]\right]+\cdots
\ee
我们得到
\be
e^{i\hat{H}t}a_\vec{p}e^{-i\hat{H}t} = a_\vec{p}e^{-iE_\vec{p} t}
\ee
\end{frame}
\begin{frame}\frametitle{最终形式}
最终我们得到
\be
\hat{\phi}(\vec{x}) = \int \frac{d^3p}{(2\pi)^{\frac32}}\frac{1}{\sqrt{2E_\vec{p}}}(\hat{a}_\vec{p}e^{-ip\cdot x}+\hat{a}^\dagger_\vec{p}e^{ip\cdot x})
\ee
另外从对易关系我们可以得到升降算符的对易
\end{frame}
\begin{frame}\frametitle{哈密顿量}
将上面的表达式带入到哈密顿量中
\be
H = \frac{1}{2} \int d^3x \dot{\phi}^2 + (\nabla \phi)^2+m^2 \phi^2
\ee
进行一顿计算后得到
\end{frame}
\end{document}