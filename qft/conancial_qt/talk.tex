\documentclass[12pt]{beamer}
\input{macros.tex}
\graphicspath{{figures/}}


\newcommand{\field}{\mathscr{F}}

\newcommand{\reals}{\mathbb{R}}
\newcommand{\complexs}{\mathbb{C}}
\newcommand{\ints}{\mathbb{Z}}
%\newcommand{\dim}{\mathrm{dim\ }}
\newcommand{\diag}{\mathrm{diag \ }}
\newcommand{\up}{\uparrow}
\newcommand{\down}{\downarrow}
\newcommand{\su}{\mathfrak{su}}
\newcommand{\so}{\mathfrak{so}}
\newcommand{\tr}{\mathrm{tr\ }}
\newcommand{\card}{\mathrm{card \ }}
\newcommand{\lag}{\mathcal{L}}
\newcommand{\hamiltonian}{\mathcal{H}}
\newcommand{\op}{\mathcal{O}}
\newtheorem{thm}{定理}[section]
\newtheorem{axm}{公理}[section]
\newtheorem{dfn}{定义}[section]
\newtheorem{experience}{经验}[section]


%\cpic{<尺寸>}{<文件名>}}用于生成居中的图片。
\newcommand{\cpic}[2]{
\begin{center}
\includegraphics[scale=#1]{#2}
\end{center}
}

%\cpicn{<尺寸>}{<文件名>}{<注释>}用于生成居中且带有注释的图片,其label为图片名。
\newcommand{\cpicn}[3]
{
\begin{figure}[h!]
\cpic{#1}{#2}
\caption{#3\label{#2}}
\end{figure}
}

\title{Quantum Field Theory\\ 正则量子化}
  \author{Haoting Xu}
  \date{\today}


\begin{document}

\begin{frame}
 
\maketitle

%\vskip 0.2in
\begin{center}
xuht9@mail2.sysu.edu.cn
\vskip 0.1in
{\tiny \url{https://github.com/HaotingXu/seminar_lec} }\\
\end{center}
\end{frame}
\section{单粒子量子力学的终结}
\secpage{单粒子量子力学的终结}{$P\simeq e^{-m|x|}$}

\begin{frame}\frametitle{海森堡绘景}
薛定谔绘景:算符不随时间改变,态随时间改变。
\be
|\psi (t)\rangle_S = e^{-i\hat{H}t}|\psi(0)\rangle_S
\ee 
海森堡绘景:算符随时间改变,态不随时间改变。为了保持算符平均值不变,定义
\be
\hat{\op}_H = e^{-i\hat{H}t} \hat{\op}_S e^{i\hat{H}t}
\ee
利用$i\partial_t |\psi(t)\rangle_S = \hat{H}|\psi(0)\rangle_S$,可以得到算符的演化方程
\be
\frac{d\hat{\op}_H(t)}{dt} = -i\left[\hat{\op}_H(t),\hat{H}\right] 
\ee
\end{frame}
\begin{frame}\frametitle{单粒子量子力学的终结}
我们现在要试图把单粒子量子力学和狭义相对论结合起来,即简单的设想$E_{\vec{p}} = \sqrt{\vec{p}^2+m^2}$,我们采取归一化$\langle \vec{x}|\vec{p}\rangle$。狭义相对论要求无法和具有类空距离的事件沟通,即要求两个态不能相互跃迁。因此我们来计算
\be
\mathcal{A} = \langle \vec{x}| e^{-i\hat{H}t}|\vec{x}=0\rangle
\ee
利用套路,有
\bea
\langle \vec{x}| e^{-i\hat{H}t}|\vec{x}=0\rangle &=& \int d^3 p \langle \vec{x}|e^{-i\hat{H}t}|\vec{p}\rangle\langle\vec{p}|\vec{x}=0\rangle\\
&=& \int d^3 p \frac{1}{(2\pi)^3}e^{i\vec{p}\cdot\vec{x}}e^{-iE_pt}
\eea
\end{frame}
\begin{frame}{进行积分}
一阵优秀的化直角坐标为球坐标的操作后,得到
\be
\mathcal{A} = \frac{-i}{(2\pi)^2 x} \int_{-\infty}^{\infty} dp p e^{ipx} e^{-it\sqrt{p^2+m^2}}
\ee
\end{frame}

\end{document}
