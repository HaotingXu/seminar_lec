\documentclass[11pt]{beamer}
\input{macros.tex}
\title{Cosmology}
  \author{Haoting}
  \date{\today}
\begin{document}

\begin{frame}
\begin{center}
\begin{Large}
 {\bf I}ntroductory  {\bf C}osmology 
{\vskip 0.1in}
Lecture 02-The Dynamics of Spacetime
\end{Large}
\end{center}
\vskip 0.1in
\begin{center}
Haoting Xu
\vskip 0.1in
xuht9@mail2.sysu.edu.cn
\vskip 0.1in
{\tiny \url{https://github.com/HaotingXu/seminar_lec/} }\\
\end{center}
\end{frame}

\section{Intro}
\begin{frame}\frametitle{宇宙如何膨胀}
上一讲我们提到,在宇宙学中,时空几何使用FRW度规来描述。
\be 
ds^2 = -c^2dt^2+ a^2(t) \lmbk{\frac{1}{1-kr^2/R^2}dr^2+r^2\lbk{d\theta^2+\sin^2\theta d\phi^2}}
\ee 
这只是一个对均匀各项同性的宇宙学原理的数学表述,并没有任何物理。物理是来决定这个度规的动力学,即$a(t)$如何演化。这一讲我们就来求解$a(t)$如何随时间演化。
\end{frame}
\begin{frame}\frametitle{广义相对论}
决定$a(t)$如何演化的物理当然是广义相对论。广义相对论可以简单理解为

Spacetime tells matter how to move, matter tells space how to curve. 

于是,几何的部分我们已经有了,而另一部分我们要解决的就是物质,或者更精确地说,物质的能量动量张量。
\end{frame}
\section{Perfect Fluids}
\secpage{Perfect Fluids}{$\rho(t) = \rho_0 a^{-3(1+w)}$}
\begin{frame}\frametitle{理想流体}
对于我们人类来说,一个星系的体积十分巨大。但是如果我们从大尺度来思考问题,每个星系实际上相当于宇宙流体的一个小原子。

对于理想流体我们只关心两件事情:第一个是流体的能量密度$\rho(t)$(在相对论的文本中,质量和能量不分家),第二个是流体的压强$P(t)$。
\end{frame}
\begin{frame}\frametitle{研究对象}
在平衡态的情况下,$P(\rho)$被称作状态方程。我们主要关心三种东西的状态方程。
\begin{itemize}
	\item 非相对论气体:尘埃。
	\item 相对论气体。
	\item 暗能量。
\end{itemize}
\end{frame}
\begin{frame}\frametitle{非相对论气体}
在这里我们直接采用统计物理中的结果,如果不熟悉的话可以去看我的统计物理讲义。对于非相对论(理想)气体,压强为
\be 
P = \frac{Nk_BT}{V}  = \frac{1}{3} \frac{N}{V} m\meanvl{v^2}
\ee 
这里用到了理想气体的速度平方的平均值。因为是非相对论情况,动能远远小于静止的能量$mc^2$,代入$E\simeq mc^2$,得到
\be 
P = \frac{NE}{V}\frac{\meanvl{v^2}}{c^2} \simeq 0
\ee 
所以我们平时讲的那点压强,在宇宙学尺度看来,几乎为$0$。
\end{frame}
\begin{frame}\frametitle{相对论气体}
对于极端相对论情况,我们在统计物理中推导过,压强和能量密度的关系为
\be 
P = \frac{1}{3} \rho 
\ee 
大部分的状态方程都有如下形式
\be 
P = w\rho 
\ee 
我们已经看到非相对论气体$w=0$,相对论气体$w = 1/3$。我们以后将会看到一些不可思议的例子。
\end{frame}
\begin{frame}\frametitle{对$w$的一个理论上的限制}
可以从流体力学中推导出来,介质中的声速满足如下关系
\be 
c_s^2 =c^2 \frac{dP}{d\rho}
\ee 
这告诉我们$w \le 1$,因为声速不能超过光速。$w$如果是负的,这时候声速$c_s$变成一个虚数,这意味着在这种介质中,声波不可能传播,因为它会指数衰减。
\end{frame}
\begin{frame}{连续性方程(4-动量流守恒方程)}
下面介绍一个决定物质的能量密度如何演化的方程。这个方程的严谨版本需要从广义相对论出发。我们之后会介绍一下这个思想。下面我们给出非严谨版本的推导,但是这个经典的随便的推导竟然和严谨的一样。考虑没有热量的热力学第一定律
\be 
dE = -p dV
\ee 
则有
\be 
\frac{dE}{dt} = -p \frac{dV}{dt}
\ee
假设今天这团东西的体积是$V_0$,那么在任何时间有$V = a^3(t)V_0$,$E = \rho a^3(t) V_0$。直接将这两个式子代入上式,得到
\be 
\dot{\rho} a^3 + 3\rho a^2 \dot{a} = -3pa^2 \dot{a}
\ee 
利用哈勃参数的定义,有
\be 
\dot{\rho} + 3H(\rho+P) =0 
\ee 
\end{frame}
\begin{frame}\frametitle{求解物质的演化}
把一般的状态方程$P = w\rho$代入上式,得到
\be 
\frac{\dot{\rho}}{\rho} = -3(1+w) \frac{\dot{a}}{a}
\ee 
直接求解,得到
\be 
\rho (t) = \rho_0 a^{-3(1+w)}
\ee 
其中$\rho_0$是今天的能量密度。值得记住两个例子,对于尘埃,有
\be 
\rho_m \sim \frac{1}{a^3}
\ee 
这和我们瞎猜的差不多一样。再看辐射
\be 
\rho_r \sim \frac{1}{a^4}
\ee 
这也和我们瞎猜的差不多一样,相比于尘埃,多了一个$1/a$因子,这恰好就是红移。
\end{frame}
\section{Friedmann Equation}
\secpage{Friedmann Equation}{$\lbk{\frac{\dot{a}}{a}}^2 = \frac{8\pi G}{3c^2}\rho - \frac{kc^2}{R^2a^2}$}
\begin{frame}\frametitle{Friedmann 方程}
重复一下我们的图像:我们考虑的是一个均匀(里面的物质是均匀的)且各向同性的宇宙。在FRW度规下和上面所述的物质的背景下,爱因斯坦方程就变成了
\be 
H^2 \equiv \fbk{\dot{a}}{a}^2 = \frac{8\pi G}{3c^2} \rho - \frac{kc^2}{R^2a}^2
\ee 
其中$k$是空间曲率,$G = 6.67\times 10^{-11} \mathrm{m^3 Kg^{-1} s^{-2}}$。因为我们没学过广义相对论,所以这部分推导暂时没法推。但是,出乎意料的是,如果做一些错误的(不完全符合)图像的假设,用牛顿引力也能推导出上面的方程。但是读者应该牢记,这个推导是“假”的。有关正确的推导,大家可以去看Dodelson的书。
\end{frame}
\begin{frame}\frametitle{求解宇宙}
最后,有了Friedmann方程和物质的演化方程(动力学方程)和物质的状态方程,就可以求解宇宙是如何演化的了。

这里就和静电场非常相似,要求解一个具体问题,首先要知道一个问题的源和电场遵从的方程(泊松方程)。
\end{frame}
\begin{frame}\frametitle{引力场}
物理学家很喜欢“场”的概念。干什么都要先定义一波场。类比于静电场,我们定义引力场,用一个引力势来描述
\be 
\vec{F} = -m\nabla \Phi  = - \int_{V^{\prime}} (\rho dV^{\prime}) \frac{Gm}{|\vec{x}-\vec{x}^{\prime}|^2}
\ee 
其中上式最后一个等号用到了牛顿万有引力定律。直接对上式求散度,则有
\bea 
-\nabla \cdot (\nabla \Phi) &=& -\nabla \cdot \lbk{\int (\rho dV^{\prime}) \frac{G}{|\vec{x}-\vec{x}^{\prime}|^2}}\\
&=& -\int (\rho dV^{\prime}) G 4\pi \delta^{(3)} (vec{x}-\vec{x}^{\prime}) \\
&=& -4\pi G\rho (\vec{x})
\eea 
这就是引力场的泊松方程。最后,我们把$\rho$换成能量密度(原来是质量密度),则有
\be 
\nabla^2 \Phi = \frac{4\pi G}{c^2} \rho 
\ee 
\end{frame}
\begin{frame}\frametitle{求解场}
我们考虑如下模型,整个宇宙就是一个大球,取$m$为大球边缘的一个小质量元,设整个大球的质量为$M(r) = \rho 4\pi r^3/3$。我们假设它是不变的。对泊松方程积分一次得到高斯定理
\be 
\int_V \nabla \cdot (\nabla \Phi) dV = \int_S (\nabla \Phi)\cdot d\vec{S} = (\nabla \Phi)_r \cdot 4\pi r^2= \int_V \frac{4\pi G}{c^2} \rho dV
\ee 
求解得到
\be 
(\nabla \Phi)_r = \frac{GM(r)}{r^2}
\ee 
于是我们有
\be 
m\ddot{r} = -\frac{GMm}{r^2}
\ee 
\end{frame}
\begin{frame}\frametitle{对动力学方程积分}
把上式对时间积分一次,得到
\be 
\int m\ddot{r}\dot{r}dt = - \int \frac{GMm}{r^2} \dot{r}dt 
\ee 
第一项分布积分,得到
\be 
\frac{1}{2} m\dot{r}^2 - \frac{GMm}{r} =E
\ee
上式就是大名鼎鼎的能量守恒定律。我认为上面这一套是物理系学生的看家本领。
\end{frame}
\begin{frame}\frametitle{Friedmann方程}
根据之前的FRW度规,代入$r= a(t)r_0$,我们定义
\be 
C = -\frac{2E}{r_0^2}
\ee 
得到Friedmann方程
\be 
\fbk{\dot{a}}{a} = \frac{8\pi G}{c^2} -\frac{C}{a^2}
\ee 
\end{frame}
\begin{frame}\frametitle{曲率}
在这里我们强行说$ C =  k^2c^2/R^2$。这是牛顿力学无法理解的。但是可以从牛顿力学有一个很好的类比。
\begin{itemize}
\item 当$C<0$时,这意味着$E>0$,而我们知道在两体运动当中,$E>0$情况对应于双曲线。双曲线就意味着$k$为负,是负曲率空间。我们将在以后看到,负曲率的宇宙的$a(t)$是大概率发散的。
\item 当$C>0$时,这意味着$E<0$,而在牛顿力学,$E<0$对应着一条闭合曲线。而我们以后会看到,正曲率的宇宙大概率是收敛的。
\end{itemize}
\end{frame}
\begin{frame}\frametitle{错误的图像给出了正确的结果}
注意我们上面推导中的图像是错误的,宇宙不是一个巨大的物质球。而且这样宇宙膨胀的图像也是错误的,宇宙并不是以某一点为中心而膨胀,而是宇宙中有无限的空间,空间中有一个个网格,是那个网格(以$a$来表示尺度)在膨胀。一个星系的共动坐标变化并不是“宇宙”意义上的膨胀。事实上从哈勃图我们也能发现,星系的共动坐标变化相对于宇宙膨胀的速度是几乎可以忽略不计的。
\end{frame}


\end{document}
